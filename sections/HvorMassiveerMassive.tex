\chapter{Hvor Massive er Massive?}
Et af kendetegnene ved et MMO er antallet af spillere, som kan spille et spil på samme tid, i det samme miljø, men hvor er grænsen for, hvornår et spil går fra at være et multiplayer spil, til at være et massive multiplayer spil? Hvis man ser på spil, som har rigtig mange spillere, som for eksempel League of Legends(LoL)\cite{LoL} eller Counter Strike(CS)\cite{CS}, så mener vi ikke, baseret på vores definition af hvad et MMO er, at disse to spilgenre eksempler (henholdsvis MOBA og FPS) kan gå inden for MMO-genren. Grunden hertil er, at selvom der er mulighed for, at kunne spille et spil med alle andre spillere, så skal man lave et seperat miljø (verden), som kun er tilgængeligt for bestemte og/eller et mindre antal af spillere. Eftersom begge spil (LoL og CS) har et stort community af spillere og under definitionen af massive \cite{Statistics}, er antallet af spillere og dermed godt kan gå ind under MMO, baseret på det grundlag , så vælger vi at frasortere disse og fokusere på MMO-spil som holder sig inden for vores definition.\\
\\
Vores konklusion på emnet er, hvis man er mange som kan spille sammen, på samme server(verden/miljø), vil dette kunne kaldes for et MMO.