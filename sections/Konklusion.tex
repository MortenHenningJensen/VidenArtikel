\cfoot{\page\textbackslash \totalp} %Skal vise x antal sider af y
\chapter{Konklusion}
I forhold til de problemstillinger der er ved at holde driften i et MMO, hvor der hele tiden kan være behov for ændringer og vedligeholdelse af forskellige dele af spillet vil en udviklingsmetode, der er flexible, være bedre egnet end en udviklingsmetode, der har faste rammer.

Efter vores research omkring emnet, mener vi at Scrum er den bedste måde at vedligeholde et spil på. Hvis man kigger på Scrum er det hele nemlig sat op i sprints, som ville være fordelagtige i vedligeholdelsen, da man kan planlægge fremad, med hvad der skal laves på eventuelle expansions, eller hvad der skal fixes i det spil man allerede har. På denne måde kan man også informere communitiet om det man laver, imens man laver det, og derved kan man altid få feedback på det man laver. Da scrum også er en agil metode kan der komme løbende ændringer og dette er optimalt for vedligeholdelsen af et produkt. Samtidig er det muligt at kunne reagere på hvad folk har af input og lave om på ting, som måske troede ville virke, eller selv synes var en god idé, men viste sig ikke at være så god når det kom til stykket.

Vi mener derimod også at vandfaldsmodellen ville være den mindst effektive til at vedligeholde et spil, da hvis man opdager fejl i det endelige produkt, ville man skulle gå helt tilbage til designfasen, og så tjekke op derfra om der er lavet fejl, og rette det trin for trin, i stedet for at kunne hoppe direkte ind i koden og rette den linje kode, der kunne give fejlen.

Med henblik på vedligeholdelse, er det situations bestemt hvilken metode der vil være bedst at bruge. For eksempel hvis man har en klasse, der kunne være en spilbar Race eller et våben, som er meget bedre end de andre alternativer i spillet, kan det være mere relevant at bruge vandfaldsmodellen, da den går helt tilbage til analyse- og designfasen og tager hele problematikken op igen. Hvorimod en Scrum metode ligger an til meget mere test og måske først ville gå ind og lave små ændringer og teste det igen, til problemet er løst.

Alle udviklingsmetoderne, som er gennemgået i denne artikel har alle sine fordele og ulemper, som skal tages hensyn til, når man vælger hvilken man vil gå ud fra når man ser hvad problemstillingen er. 
